The Standard Model is widely hailed as the most precisely tested model ever. Measurements
of the electroweak sector at LEP agree astoundingly well with predictions
from theory, and the recent discovery of the Higgs boson at the LHC appears to
have completed the picture.
Unfortunately the Standard Model has some very large, very obvious flaws.
Firstly, it makes no statement about gravity despite it being one of the largest
driving forces of the formation of galaxies, stars, and planets. In addition, measurements
of the rotational velocities of galaxies have revealed that there is very
likely more matter in the universe than we can see. The Standard Model does not
make any predictions for what this “dark matter” could be made of. In fact, recent
measurements of distant supernovae suggest that the expansion of the universe is
accelerating, requiring some form of energy source - this “dark energy” is also not
predicted anywhere in the Standard Model. The final large flaw is the models current
inability to provide the drastic matter-antimatter asymmetry which is seen in
the universe today.
The discovery of neutrino oscillations in the early 2000’s [1, 2], described in
chapter 2 has shown one of the first glimpses of new physics beyond the Standard
Model, as it demonstrates that neutrinos have mass (though it doesn’t tell us exactly
what that mass is). This raises new and interesting questions such as what is the
neutrinos mass, and why is it so small?
Neutrino oscillations allow the possibility of CP-violation in the lepton sector
which is required by many new models which generate the matter-antimatter asymmetry
in the universe. In addition, some proposals of additional “sterile” neutrinos
can function as dark matter candidates. These sterile neutrinos, if they exist, may
be indirectly discovered in oscillation experiments.
