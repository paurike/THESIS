Much of modern physics is supported by a bedrock principle which continues to hold true without exception and which guides physicists' intuition, perhaps more so than any other fundamental law - the conservation of energy. However, around the first quarter of the previous century it looked like it might have to be abandoned in the light of new experimental findings despite most physicists' natural resistance to the idea. The quantum mechanics revolution had just upturned many of the firmly held conceptions about the fundamental nature of reality and so it might not have seemed like such a stretch of the imagination that even something as difficult to fathom as non-conservation of energy might be lurking in the puzzling world of fundamental particles, that was only just being explored through a new model, established by Niels Bohr's resounding success in explaining the hydrogen atom and inspired by Ernest Rutherford's discovery of the existence of atomic nuclei before him. The troubling observation that the energy of an electron produced by $\beta$-decay is on a continuos spectrum rather than at a fixed value as two-body kinematics dictates, lead Bohr to speculate that energy on the atomic scale might only appear conserved when averaged over many events. Wolfgang Pauli, not without reason nicknamed \textit{the Scourge of God} by his colleagues\cite{Faust}, proposed something arguably bolder. The postulation of new particles or forms of matter has earned its place in the standard repertoire of the modern theoretician as an elegant tool to solve many otherwise inexplicable problems\footnote{with the convenient prefix 'dark' reserved for the most confounding specimens}, experimental validation is forever confirming the addition of new particles to the canon and the rich mathematical structure revealed by the systematic grouping of their properties has spurred on the search for what is called for by symmetry considerations. However, in a world thought to be made up entirely of electrons and protons, the arrival of the neutrino (then dubbed 'neutron') must have come as a shock, and so it is only fitting that Pauli's first mention of its possible existence was delivered as part of an apology. A ball in Vienna prevented him from personally explaining his startling proposition - that he called a 'desperate remedy'\footnote{Later in life he also refered to it as "that foolishly behaving foolish child of my life's 1930-31 crisis"} - to his 'radioactive' colleagues gathered in Tuebingen in the winter of 1930.     