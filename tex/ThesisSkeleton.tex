\documentclass[a4paper, 11pt]{report}

\usepackage{amsmath}
\usepackage{graphicx}
\usepackage{caption}
\usepackage{subcaption}
\usepackage{geometry}
\usepackage{../titlepage/warwickthesis}
\usepackage{enumerate}

\usepackage[onehalfspacing]{setspace}

%\geometry{
%	a4paper,
%	left=4cm,
%	top=4cm,
%	right=27mm,
%	bottom=44mm,	
%	}
	
\usepackage{hyperref}
\hypersetup{
    colorlinks,
    citecolor=black,
    filecolor=black,
    linkcolor=black,
    urlcolor=black
}	


\date{\today}
\author{Paula Friederike Denner}
\title{A real good Title}

\newlength\longest
\leftchapter  

\begin{document}

%\thesiscopyrightpage

\thesistitlepage




\clearpage

\pagestyle{empty}
\null\vfill

\settowidth\longest{\Large\itshape \hspace{3cm}Sometimes, if you pay real close attention to the pebbles ;}
\parbox{\longest}{%
  \raggedright{\large\itshape%
  
  \hspace{1cm}"You know, Sir, sometimes I think\\ 
  \hspace{1cm}there's a great ocean of thruth out there\\
  \hspace{1cm}and I'm just sitting on the beach\\
  \hspace{1cm}playing with...with stones."\\
  \par\bigskip
  }
  \hspace{8cm} \large\MakeUppercase{Ponder Stibbons}\par%
  \vspace{2cm}
  
  \raggedright{\large\itshape%
  \hspace{1cm}"Sometimes, if you pay real close attention to the pebbles \\ 
  \hspace{1cm}you find out about the ocean."\\ 
  \par\bigskip
  }   
  \vspace{1cm}
  \hspace{8cm} \large\MakeUppercase{Terry Pratchett}\par%
}
\vfill\vfill
\clearpage



\pagestyle{plain}

\pagenumbering{roman}

\tableofcontents{}


\listoffigures{}
%\addcontentsline{toc}{chapter}{List of Figures}


\listoftables{}
%\addcontentsline{toc}{chapter}{List of Tables}

\clearpage

\section*{Acknowledgments}

These are my Acknowledgments.
I thank bla bla bla...
Lorem ipsum dolor sit amet, consetetur sadipscing elitr, sed diam nonumy eirmod tempor invidunt ut labore et dolore magna aliquyam erat, sed diam voluptua. At vero eos et accusam et justo duo dolores et ea rebum. Stet clita kasd gubergren, no sea takimata sanctus est Lorem ipsum dolor sit amet. Lorem ipsum dolor sit amet, consetetur sadipscing elitr, sed diam nonumy eirmod tempor invidunt ut labore et dolore magna aliquyam erat, sed diam voluptua. At vero eos et accusam et justo duo dolores et ea rebum. Stet clita kasd gubergren, no sea takimata sanctus est Lorem ipsum dolor sit amet. Lorem ipsum dolor sit amet, consetetur sadipscing elitr, sed diam nonumy eirmod tempor invidunt ut labore et dolore magna aliquyam erat, sed diam voluptua. At vero eos et accusam et justo duo dolores et ea rebum. Stet clita kasd gubergren, no sea takimata sanctus est Lorem ipsum dolor sit amet.   

Duis autem vel eum iriure dolor in hendrerit in vulputate velit esse molestie consequat, vel illum dolore eu feugiat nulla facilisis at vero eros et accumsan et iusto odio dignissim qui blandit praesent luptatum zzril delenit augue duis dolore te feugait nulla facilisi. Lorem ipsum dolor sit amet, consectetuer adipiscing elit, sed diam nonummy nibh euismod tincidunt ut laoreet dolore magna aliquam erat volutpat.   

Ut wisi enim ad minim veniam, quis nostrud exerci tation ullamcorper suscipit lobortis nisl ut aliquip ex ea commodo consequat. Duis autem vel eum iriure dolor in hendrerit in vulputate velit esse molestie consequat, vel illum dolore eu feugiat nulla facilisis at vero eros et accumsan et iusto odio dignissim qui blandit praesent luptatum zzril delenit augue duis dolore te feugait nulla facilisi.   

Nam liber tempor cum soluta nobis eleifend option congue nihil imperdiet doming id quod mazim placerat facer possim assum. Lorem ipsum dolor sit amet, consectetuer adipiscing elit, sed diam nonummy nibh euismod tincidunt ut laoreet dolore magna 

Ut wisi enim ad minim veniam, quis nostrud exerci tation ullamcorper suscipit lobortis nisl ut aliquip ex ea commodo consequat. Duis autem vel eum iriure dolor in hendrerit in vulputate velit esse molestie consequat, vel illum dolore eu feugiat nulla facilisis at vero eros et accumsan et iusto odio dignissim qui blandit praesent luptatum zzril delenit augue duis dolore te feugait nulla facilisi.   

Nam liber tempor cum soluta nobis eleifend option congue nihil imperdiet doming id quod mazim placerat facer possim assum. Lorem ipsum dolor sit amet, consectetuer adipiscing elit, sed diam nonummy nibh euismod tincidunt ut laoreet.



\addcontentsline{toc}{chapter}{Acknowledgments}

\clearpage

\section*{Declarations}

These are my Declarations...Lorem ipsum dolor sit amet, consetetur sadipscing elitr, sed diam nonumy eirmod tempor invidunt ut labore et dolore magna aliquyam erat, sed diam voluptua. At vero eos et accusam et justo duo dolores et ea rebum. Stet clita kasd gubergren, no sea takimata sanctus est Lorem ipsum dolor sit amet. Lorem ipsum dolor sit amet, consetetur sadipscing elitr, sed diam nonumy eirmod tempor invidunt ut labore et dolore magna aliquyam erat, sed diam voluptua. At vero eos et accusam et justo duo dolores et ea rebum. Stet clita kasd gubergren, no sea takimata sanctus est Lorem ipsum dolor sit amet. Lorem ipsum dolor sit amet, consetetur sadipscing elitr, sed diam nonumy eirmod tempor invidunt ut labore et dolore magna aliquyam erat, sed diam voluptua. At vero eos et accusam et justo duo dolores et ea rebum. Stet clita kasd gubergren, no sea takimata sanctus est Lorem ipsum dolor sit amet.   

Duis autem vel eum iriure dolor in hendrerit in vulputate velit esse molestie consequat, vel illum dolore eu feugiat nulla facilisis at vero eros et accumsan et iusto odio dignissim qui blandit praesent luptatum zzril delenit augue duis dolore te feugait nulla facilisi. Lorem ipsum dolor sit amet, consectetuer adipiscing elit, sed diam nonummy nibh euismod tincidunt ut laoreet dolore magna aliquam erat volutpat.   

Ut wisi enim ad minim veniam, quis nostrud exerci tation ullamcorper suscipit lobortis nisl ut aliquip ex ea commodo consequat. Duis autem vel eum iriure dolor in hendrerit in vulputate velit esse molestie consequat, vel illum dolore eu feugiat nulla facilisis at vero eros et accumsan et iusto odio dignissim qui blandit praesent luptatum zzril delenit augue duis dolore te feugait nulla facilisi.   

Nam liber tempor cum soluta nobis eleifend option congue nihil imperdiet doming id quod mazim placerat facer possim assum. Lorem ipsum dolor sit amet, consectetuer adipiscing elit, sed diam nonummy nibh euismod tincidunt ut laoreet dolore magna 

Ut wisi enim ad minim veniam, quis nostrud exerci tation ullamcorper suscipit lobortis nisl ut aliquip ex ea commodo consequat. Duis autem vel eum iriure dolor in hendrerit in vulputate velit esse molestie consequat, vel illum dolore eu feugiat nulla facilisis at vero eros et accumsan et iusto odio dignissim qui blandit praesent luptatum zzril delenit augue duis dolore te feugait nulla facilisi.   

Nam liber tempor cum soluta nobis eleifend option congue nihil imperdiet doming id quod mazim placerat facer possim assum. Lorem ipsum dolor sit amet, consectetuer adipiscing elit, sed diam.
\addcontentsline{toc}{chapter}{Declarations}

\clearpage
\null\vfill

{\centering
\section*{Abstract}
}
Nulla suscipit orci sed lectus pharetra, sed scelerisque tellus molestie. Sed vitae metus aliquam, sollicitudin urna ullamcorper, dapibus felis. In nec massa vitae eros porttitor dapibus vel nec ligula. Nunc malesuada, dui non vehicula posuere, eros eros aliquet turpis, quis vulputate lorem mauris vitae ante. Aenean mattis nibh vitae est dignissim dictum. Curabitur pellentesque mollis accumsan. Vestibulum sit amet arcu non massa cursus rutrum convallis sed nulla. Nunc vestibulum volutpat erat, id porttitor metus pulvinar nec. Praesent nec dui massa. Nulla elementum justo ex, nec ultricies nisl porttitor a. Cras placerat tincidunt sollicitudin.

Morbi ultrices vitae lorem non varius. Nulla rutrum nisl ut gravida sagittis. Nulla nec mi elit. Curabitur ac quam felis. Morbi tempor, dui eget euismod vehicula, dolor ipsum feugiat nibh, vitae scelerisque erat erat vitae metus. Nunc ornare, quam in malesuada sollicitudin, nisl lorem vehicula mi, quis ornare elit augue vel ex. Nunc in tempus ante.

Etiam blandit dolor ac dolor tincidunt, in ultrices urna suscipit. Aliquam erat volutpat. Cras in tempor nisl. Sed rhoncus imperdiet cursus. Mauris ex magna, posuere a lacus nec, tempus ullamcorper massa. Morbi ac tellus sed metus iaculis luctus vel non nulla. Vivamus accumsan, urna quis scelerisque cursus, nulla lectus tristique ex, et posuere elit lorem in ligula. Aliquam fermentum magna vel libero luctus mollis. Nunc vitae est id lectus volutpat lacinia sit amet ac tortor. Quisque id libero laoreet, sodales libero vel, aliquet quam.


\addcontentsline{toc}{chapter}{Abstract}

\vfill\vfill
\clearpage


\pagenumbering{arabic}

\chapter{Introduction}
The Standard Model is widely hailed as the most precisely tested model ever. Measurements
of the electroweak sector at LEP agree astoundingly well with predictions
from theory, and the recent discovery of the Higgs boson at the LHC appears to
have completed the picture.
Unfortunately the Standard Model has some very large, very obvious flaws.
Firstly, it makes no statement about gravity despite it being one of the largest
driving forces of the formation of galaxies, stars, and planets. In addition, measurements
of the rotational velocities of galaxies have revealed that there is very
likely more matter in the universe than we can see. The Standard Model does not
make any predictions for what this “dark matter” could be made of. In fact, recent
measurements of distant supernovae suggest that the expansion of the universe is
accelerating, requiring some form of energy source - this “dark energy” is also not
predicted anywhere in the Standard Model. The final large flaw is the models current
inability to provide the drastic matter-antimatter asymmetry which is seen in
the universe today.
The discovery of neutrino oscillations in the early 2000’s [1, 2], described in
chapter 2 has shown one of the first glimpses of new physics beyond the Standard
Model, as it demonstrates that neutrinos have mass (though it doesn’t tell us exactly
what that mass is). This raises new and interesting questions such as what is the
neutrinos mass, and why is it so small?
Neutrino oscillations allow the possibility of CP-violation in the lepton sector
which is required by many new models which generate the matter-antimatter asymmetry
in the universe. In addition, some proposals of additional “sterile” neutrinos
can function as dark matter candidates. These sterile neutrinos, if they exist, may
be indirectly discovered in oscillation experiments.


Hello This is going to be my Thesis\cite{aBook}

Lorem ipsum dolor sit amet, consectetur adipiscing elit. Aliquam facilisis sollicitudin massa non aliquam. Aliquam neque augue, semper vel pharetra ac, facilisis in enim. Morbi eget consectetur libero. Proin pulvinar maximus fringilla. Pellentesque scelerisque hendrerit nisl ac pretium. Suspendisse potenti. Suspendisse eget varius nulla. Curabitur id interdum dolor. Duis leo diam, fermentum a bibendum quis, facilisis ac ante. Maecenas nunc enim, finibus in dolor vel, tincidunt malesuada augue.

Fusce dictum vitae tellus volutpat tempor. Pellentesque euismod sollicitudin magna non pulvinar. Praesent ut feugiat metus. Nulla vitae tortor euismod eros luctus tincidunt sed vitae elit. Fusce augue ante, euismod eget erat eu, blandit egestas diam. Nunc eleifend, nisi non lobortis tincidunt, arcu velit sodales augue, vitae tempor ipsum urna eu nisi. Aenean sit amet ullamcorper elit. Donec metus tortor, accumsan consequat mollis id, malesuada sit amet dolor. Pellentesque risus est, consectetur eu ante vel, varius laoreet sapien.

Nulla suscipit orci sed lectus pharetra, sed scelerisque tellus molestie. Sed vitae metus aliquam, sollicitudin urna ullamcorper, dapibus felis. In nec massa vitae eros porttitor dapibus vel nec ligula. Nunc malesuada, dui non vehicula posuere, eros eros aliquet turpis, quis vulputate lorem mauris vitae ante. Aenean mattis nibh vitae est dignissim dictum. Curabitur pellentesque mollis accumsan. Vestibulum sit amet arcu non massa cursus rutrum convallis sed nulla. Nunc vestibulum volutpat erat, id porttitor metus pulvinar nec. Praesent nec dui massa. Nulla elementum justo ex, nec ultricies nisl porttitor a. Cras placerat tincidunt sollicitudin.

Morbi ultrices vitae lorem non varius. Nulla rutrum nisl ut gravida sagittis. Nulla nec mi elit. Curabitur ac quam felis. Morbi tempor, dui eget euismod vehicula, dolor ipsum feugiat nibh, vitae scelerisque erat erat vitae metus. Nunc ornare, quam in malesuada sollicitudin, nisl lorem vehicula mi, quis ornare elit augue vel ex. Nunc in tempus ante.

Etiam blandit dolor ac dolor tincidunt, in ultrices urna suscipit. Aliquam erat volutpat. Cras in tempor nisl. Sed rhoncus imperdiet cursus. Mauris ex magna, posuere a lacus nec, tempus ullamcorper massa. Morbi ac tellus sed metus iaculis luctus vel non nulla. Vivamus accumsan, urna quis scelerisque cursus, nulla lectus tristique ex, et posuere elit lorem in ligula. Aliquam fermentum magna vel libero luctus mollis. Nunc vitae est id lectus volutpat lacinia sit amet ac tortor. Quisque id libero laoreet, sodales libero vel, aliquet quam.

\chapter{Neutrino Physics}
\section{A Brief History of Neutrino Physics}

\section{Neutrinos in the Standard Model}
\section{Neutrino Oscillations and Neutrino Mass}
\subsection{Oscillation Phenomenology}
\subsection{Matter effects}
\subsection{Absolute Neutrino Mass}
\subsection{The Mass Hierarchy}
\subsection{CP-violation}
\subsection{The State of the Field}

\section{Neutrino Interactions}
\subsection{Neutrino Cross-Section Models}

\chapter{Contemporary and Next Generation Experiments}
\section{Future Long Baseline Experiments}
\subsection{Hyper-Kamiokande}
\subsection{The Deep Underground Experiment (DUNE)}
\section{High Pressure Argon Gas Time Projection Chambers}
This is the DUNE Ananlysis section

\subsection{Near Detector Concept}
\subsection{Reconstructing simulated Events}
\subsection{Resolution/Sensitivity Study(?)}

\chapter{The Tokai-to-Kamioka Experiment (T2K)}
\section{Super-Kamiokande and the Near Detector (ND280)}
\section{The ND280 Software Processing Chain}
\subsection{Calibration}
\subsection{Reconstruction}
\subsection{Analysis}
This is the T2K Analysis section
\subsection{Super-Kamiokande and the Near Detector (ND280)}
\subsection{The ND280 Software Processing Chain}
\subsection{The Gas Interaction Event Selection}
\subsection{Systematic Uncertainties}
\subsection{Measurement and Result}

\chapter{Event Reconstruction with TREx}
\section{The TREx Algorithms}
\subsection{Pattern Recognition}
\subsection{Track Fitting, Matching and Merging}
\section{Adapting TREx for High Pressure Gas TPCs}
\section{Short Proton Track Optimisation}

\chapter{Neutrino Interactions on Argon Gas}
\section{The Selection}
\section{Systematic Uncertainties}
\section{Cross-section Measurement and Results}

\chapter{Conclusions}
\input{Conclusions}

\chapter*{Appendix}
\section*{Philosophical Essays}
\subsection*{What are Objects/Are there Objects?}

\addcontentsline{toc}{chapter}{Appendix}

\chapter*{Glossary of Terms and Abbreviations}
\input{Glossary.tex}
\addcontentsline{toc}{chapter}{Glossary of Terms and Abbreviations}

\addcontentsline{toc}{chapter}{Bibliography}
\bibliography{ThesisSkeleton}
\bibliographystyle{plain}

\end{document}
